%%%%%%%%%%%%%%%%%%%%%%%%%%%%%%%%%%%%%%%%%%%%%%%%%%%%%%%%%%%%%%%%%%%%%%%%%%%%%%%%

\begin{abstract}
\thispagestyle{empty}
En este Trabajo de Fin de Grado (TFG) se desarrollará un subsistema de interacción humano robot para una plataforma robótica controlada desde un teléfono inteligente (smartphone) basado en Android, denominada ROBOBO.

ROBOBO es el resultado de la combinación de un smartphone con una plataforma motorizada y dotada de sensores de proximidad, y que ha sido desarrollada por el GII (Grupo Integrado de Ingeniería de la UDC) con el objetivo de conseguir un robot de bajo coste, con multitud de sensores y métodos de conexión inalámbrica. Estas características, unidas a la constante evolución de los teléfonos inteligentes tanto en características como en capacidad de procesado, hacen del ROBOBO una opción interesante de cara a la investigación robótica y la educación.

El ROBOBO! Framework es el marco de trabajo empleado para el desarrollo de aplicaciones para el ROBOBO, y proporciona diferentes funcionalidades útiles a la hora del desarrollo, como la interfaz de control de la base motorizada, en forma de módulos.

El objetivo de este TFG es la expansión de dicho framework mediante el desarrollo de un subsistema de interacción entre humanos y el robot, que proporcione al programador diferentes posibilidades de interacción entre el robot y el usuario final. 
\end{abstract}

%%%%%%%%%%%%%%%%%%%%%%%%%%%%%%%%%%%%%%%%%%%%%%%%%%%%%%%%%%%%%%%%%%%%%%%%%%%%%%%%
