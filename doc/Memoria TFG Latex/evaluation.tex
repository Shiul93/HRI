\chapter{Resultados}
\label{chap:results}
\vspace{0.5cm}

%%%%%%%%%%%%%%%%%%%%%%%%%%%%%%%%%%%%%%%%%%%%%%%%%%%%%%%%%%%%%%%%%%%%%%%%%%%%%%%%
% Objetivo: Exponer los resultados objetivos del sistema                       %
%%%%%%%%%%%%%%%%%%%%%%%%%%%%%%%%%%%%%%%%%%%%%%%%%%%%%%%%%%%%%%%%%%%%%%%%%%%%%%%%

 
\lettrine{E}{n} este capítulo se expondrán los resultados y el funcionamiento de los módulos desarrollados en este trabajo mediante tres aplicaciones de ejemplo.
\section{Ejemplos de uso}
\subsection{Simon dice músical}
\subsection{Robobo Vigilante}
\subsection{Robobo Mascota}


\section{Problemas conocidos}
\begin{itemize}
	\item La tasa de refresco del módulo \textit{BasicCamera} es baja y varía mucho entre terminales móviles.
	\item El \textit{ColorDetectionModule} puede confundirse si el fondo no es homogéneo, se recomienda usar tarjetas con colores sobre fondo blanco.
	\item El \textit{EmailModule} puede causar el bloqueo de la cuenta de Gmail si esta no es configurada previamente para usar mediante IMAP.
	\item El \textit{TouchModule} requiere el paso explícito de los TouchEvents de la actividad en pantalla.
\end{itemize}
\label{sec:known_issues}
    

