\chapter{Interacción Humano Robot}
\label{chap:interaccion-humano-robot}

%%%%%%%%%%%%%%%%%%%%%%%%%%%%%%%%%%%%%%%%%%%%%%%%%%%%%%%%%%%%%%%%%%%%%%%%%%%%%%%%
% Objetivo: Exponer de qué va este proyecto, sus líneas maestras, objetivos,   %
%           etc.                                                               %
%%%%%%%%%%%%%%%%%%%%%%%%%%%%%%%%%%%%%%%%%%%%%%%%%%%%%%%%%%%%%%%%%%%%%%%%%%%%%%%%

%todo http://humanrobotinteraction.org/human-robot-interaction-a-historical-perspective-and-current-research-trends/


%Objetivos de interacción del proyecto, modelo conceptual del mismo.

\lettrine{E}{n} este capítulo se introduce el tema de la Interacción humano robot, 	y el diseño conceptual del sistema de interacción desarrollado para la plataforma ROBOBO.

\section{Introducción}
 \label{sec:hri-intro}
 
 %todo meter lo de las conferencias
 Desde finales de la década de los 90, la interacción entre humanos y robots ha cobrado importancia, creando un nuevo campo de investigación en la ciencia\cite{goodrich2007human}, la interacción humano-robot (HRI por sus siglas en inglés). El campo de la HRI busca entender, modelar y evaluar las diferentes modalidades de interacción entre las personas y los robots. La comunicación entre humanos y robots puede dividirse  en dos categorías generales:
 \begin{itemize}
 	\item Interacción remota
 	\item Interacción próxima
 \end{itemize}
 
 La interacción remota suele ser referida como control supervisado o teleoperación, dependiendo de si el robot es autónomo con supervisión de un humano, que interviene en caso de necesidad, o si el robot es controlado por el humano directamente. Este tipo de interacción puede verse en robots de tipo industrial o en vehículos autónomos, cómo los llamados Drones del ejercito.
 
 La interacción próxima es aquella en la que el robot interactúa directamente con el humano, llegando incluso a haber interacción física. Este tipo de interacción incluye elementos emotivos y sociales, y se puede encontrar en, por ejemplo, los robots asistenciales o educativos. Este tipo de interacción es la que se tratará en este trabajo, en el cual se diseñarán diferentes sistemas de interacción para la plataforma ROBOBO.
 
 \subsection{HRI en la industria}
 
 Generalmente, en los procesos industriales, la interacción entre los robots y los operadores suele ser remota, los sistemas se programan para realizar una tarea y el humano solamente interviene en caso de necesidad, sin embargo pueden darse casos de interacción próxima con los robots. Uno de estos casos podría ser el aprendizaje de tareas mediante demostración, proceso mediante el cual, un operador humano realiza una tarea, por ejemplo moviendo manualmente el brazo de un robot, para que el controlador aprenda a realizar esa tarea. Este tipo de interacción permite que los robots aprendan comportamientos de alto nivel difícilmente programables.\\
 Sin embargo a la hora de realizar tareas de forma cooperativa entre robots y humanos, la interacción cercana con robots industriales conlleva riesgos importantes, como pudo verse en el accidente del 2015\cite{vwaccident2015} en la planta de Volkswagen cerca de Kassel, Alemania, en el que un operario fué golpeado por un brazo industrial  durante su instalación, resultando en la muerte del técnico.
  \begin{figure}
	\centering
	\includegraphics[width=1\linewidth]{imagenes/robotcell.jpeg}
	\caption{Esquema de una jaula de seguridad para un robot industrial}
	\label{fig:robot-cell}
\end{figure} 
 Para evitar esta clase de accidentes, se está buscando la consciencia del entorno en los manipuladores robóticos, para poder adaptar sus reacciones al contexto actual. Este tipo de consciencia no solo disminuye los riesgos de operación, sino que también disminuye espaciales, ya que los robots no requerirían de jaulas de seguridad (figura \ref{fig:robot-cell}) , también la productividad se vería afectada positivamente, ya que tareas imposibles de realizar para un robot y para un humano individualmente, pueden llevarse a cabo mediante la llamada robótica cooperativa.
 En \textit{Cooperative Tasks between Humans and Robots in Industrial Environments}\cite{corrales2012cooperative} presentan un sistema de robótica cooperativa en el que un operador y un robot colaboran de manera cercana para llevar a cabo diferentes tareas, de manera que el robot realiza las tareas repetitivas y peligrosas, mientras que el humano lleva a cabo las tareas que requieren de cierta precisión o inteligencia con la que no cuenta el robot. En este sistema el operador lleva un traje de posicionamiento, que permite al robot conocer su posición, pudiendo así adaptar sus movimientos de manera que el humano no corra riesgos.
 
 
 
 \subsection{HRI en robots asistenciales}
 Uno de los campos en los que la interacción entre humanos y robots cobra mucha importancia es en el nicho de los robots asistenciales. Los robots asistenciales, también llamados de servicio, son definidos por la federación internacional de robótica cómo \textit{Robots que operan de forma total o semiautónoma para realizar servicios útiles para el bienestar de humanos y equipamiento, excluyendo las operaciones de manufactura}\cite{ifr-service-robots}. En esta definición se diferencia entre dos tipos de robots asistenciales:
 \begin{itemize}
 	\item Robots personales
 	\item Robots profesionales
 \end{itemize}
 
 Los robots personales son aquellos que se utilizan para labores no comerciales, generalmente por personas sin perfil técnico. Por ejemplo, sillas de ruedas eléctricas, robots de asistencia de movilidad, o aspiradoras automáticas.
 
 Los robots profesionales son aquellos utilizados para realizar tareas de asistencia en un entorno comercial, generalmente manejados y supervisados por un personal especializado. Por ejemplo robots de limpieza automatizados para zonas públicas, robots de mensajería en oficinas u hospitales, robots anti-incendios,robots quirúrgicos y de rehabilitación en hospitales o los robots terapéuticos.
 \begin{figure}
	\centering
	\includegraphics[width=0.8\linewidth]{imagenes/parorobot.JPG}
	\caption{Robot terapeutico Paro}
	\label{fig:parorobot}
\end{figure} 

 En los robots terapéuticos se pueden encontrar múltiples formas de interacción, por ejemplo, el robot Paro\cite{parorobots} (figura \ref{fig:parorobot}) es un robot terapéutico con forma de bebé foca, utilizado con éxito en terapias contra la Demencia, que busca una interacción emocional con el paciente, para ello cuenta con cinco tipos de sensores diferentes, táctiles, auditivos, de temperatura, de luz y posturales. Los pacientes realizan una interacción con el robot cómo la que tendrían con un animal, y el robot responde acorde a los estímulos que recibe. Esto, en conjunto con la forma física del robot, más semejante a un animal de peluche que a una máquina, permite al paciente desarrollar emociones.
   \begin{figure}
	\centering
	\includegraphics[width=1\linewidth]{imagenes/naorobot.jpg}
	\caption{Esquema de características del robot NAO}
	\label{fig:naorobot}
\end{figure} 
 El robot NAO\cite{naorobots}(figura \ref{fig:naorobot}) es otro de los robots que se están empleando con éxito en tareas asistenciales, tanto de manera terapéutica, como robot de relaciones publicas o para tareas de educación. El robot cuenta con una amplia variedad de sensores y actuadores que le permiten interactuar de diversas formas con el usuario:
 \begin{itemize}
 	\item Cámaras: Permiten reconocimiento de caras y procesado de imagen.
 	\item Sensores táctiles y de presión: Permiten una interacción física con el robot.
 	\item Altavoces: Permiten al robot producir diversos sonidos y hablar.
 	\item Micrófonos: Permiten reconocer habla y ubicar el origen de los sonidos espacialmente.
 	\item Sensores de distancia: Permiten detectar la distancia a los objetos.
 	\item 25 Grados de libertad: Permiten al robot interactuar fisicamente con su entorno y realizar comunicación no verbal.
 	\item Unidad de medición inercial: Permite detectar aceleraciones y giros.
 \end{itemize}
\begin{figure}
	\centering
	\includegraphics[width=1\linewidth]{imagenes/davincisystem.jpg}
	\caption{Sistema quirúrgico daVinci}
	\label{fig:davincisystem}
\end{figure} 
 
 En el caso de los robots médicos, los quirúrgicos especificamente, la interacción no suele pasar de la teleoperación del robot, es decir, el robot se controla como si fuera una extensión del cirujano. El robot más conocido en este campo es el sistema quirúrgico Da Vinci (figura \ref{fig:davincisystem}), utilizado en cirugía de precisión. En este robot el cirujano controla los diferentes brazos del aparato desde una consola que cuenta con controles con feedback háptico, es decir, el operador no mueve unicamente el brazo, sino que siente lo que hay al final del mismo, la presión ejercida y la resistencia al movimiento, dando al cirujano el tacto necesario para realizar las diferentes tareas que se realizan en una operación, como coser o cortar, de forma natural y con un alto grado de precisión.
 

 
 \subsection{HRI en robots de entretenimiento}
 Se entiende por robots de entretenimiento aquellos cuya finalidad no es más que divertir al usuario. Dentro de esta categoría podríamos incluir a los robots mascota, que generalmente realizan una interacción de alto nivel con el usuario.
 Por ejemplo, uno de los robots más relevantes en el ámbito de los robots mascota es el desarrollado por Sony, Aibo \cite{aibo} (figura \ref{fig:aiborobot}) , cuya finalidad era comportarse como un perro. En este robot podemos encontrar múltiples tipos de interacción, interacción física en forma de movimientos perrunos, reconocimiento facial a través de cámaras, a través de caricias utilizando los sensores táctiles, y en las últimas versiones del robot mediante una matriz de leds situada en la cara del aparato, que permite poner diferentes expresiones en función del "humor" del robot. Mediante todas estas capacidades motoras y sensoriales, se puede interactuar con una unidad Aibo de manera semejante a la que se tendría con un perro real.
 Otro ejemplo de mascota robótica es el Bandai SmartPet (figura \ref{fig:bandaismartpet}) , un robot pensado para utilizar junto un smartphone IPhone, que es colocado en la cabeza del robot y provee múltiples formas de interacción, reconocimiento de gestos mediante la cámara frontal, reconocimiento de sonido utilizando el micrófono y reconocimiento de gestos táctiles en la pantalla.
 
 \begin{figure}
	\centering
	\includegraphics[width=0.5\linewidth]{imagenes/aiborobot.jpg}
	\caption{Robot mascota Aibo de Sony}
	\label{fig:aiborobot}
 \end{figure}
  \begin{figure}
	\centering
	\includegraphics[width=0.5\linewidth]{imagenes/bandaismartpet.jpg}
	\caption{Robot mascota Smartpet de Bandai}
	\label{fig:bandaismartpet}
 \end{figure}

 \subsection{HRI en robots educativos}
 En  los últimos años el uso de robots con fines educativos, si bien ya eran usados de manera educativa en enseñanza superior, ha tomado impulso en la educación primaria y secundaria, apareciendo múltiples robots enfocados a este tipo de mercado. La interacción con esta clase de robots puede darse de diferentes formas según el público objetivo, dependiendo principalmente del rango de edades del mismo.

\begin{figure}
\centering
\begin{minipage}{0.45\textwidth}
\centering
\includegraphics[width=1\linewidth]{imagenes/beebot.png}
\caption{Beebot}
\label{fig:beebot}

\end{minipage}\hfill
\begin{minipage}{0.45\textwidth}
\centering
\includegraphics[width=1\linewidth]{imagenes/escornabot.jpg}

\caption{Escornabot}
\label{fig:escornabot}

\end{minipage}
\end{figure}

 Enfocados a la educación infantil nos podemos encontrar con robots cómo el BeeBot(figura \ref{fig:beebot}) o su alternativa libre, el Escornabot(figura \ref{fig:escornabot}),la finalidad de estos robots es la introducción a los niños a la programación, en forma de pensamiento secuencial, y a la resolución de problemas. En esta clase de robots la interacción está limitada a la introducción de comandos en la botonera de la parte superior del robot, que serán traducidos a movimientos del robot posteriormente.  

 En la educación primaria y secundaria los robots utilizados ya adquieren una mayor complejidad y suelen emplearse para una introducción real a la programación, generalmente utilizando lenguajes gráficos de muy alto nivel como Scratch\cite{scratch}. Uno de los robots más relevantes en este ámbito es el kit Mindstorms de Lego (figura \ref{fig:legoev3}) , que no solo permite programar el robot, sino también construirlo. El Lego Mindstorms cuenta con diferentes sensores y actuadores que ofrecen diversas maneras de interacción con el robot:
 \begin{itemize}
 	\item Motores: Permiten el movimiento del robot de manera relativamente precisa
 	\item Sensores de distancia: Permiten medir distancias y actuar en consecuencia a los datos medidos
 	\item Sensores de luz ambiente: Miden la intensidad de la luz del entorno
 	\item Sensores de color: Permiten detectar los colores básicos
 	\item Unidad inercial: Permite medir giros en el plano horizontal
 \end{itemize}
 
   \begin{figure}
	\centering
	\includegraphics[width=0.8\linewidth]{imagenes/legoev3.jpg}
	\caption{Kit básico Lego Mindstorms EV3}
	\label{fig:legoev3}
\end{figure} 

En la llamada educación especial la robótica también esta siendo empleada con éxito, por ejemplo el robot NAO (figura \ref{fig:naorobot}), del que se habló en la sección anterior, se utiliza para educar a niños con trastornos de espectro autista. El éxito de este tipo de educación viene dada debido a que la interacción con el robot, al ser programada, resulta predecible para el alumno, creando un entorno estable y pautado que resulta óptimo en este tipo de trastornos. La interacción en este caso se da en forma de movimientos preprogramados y mediante los leds de los ojos del robot, que cambian de color segun las emociones que intenta expresar el robot, lo cual ayuda al alumno a ejercitar uno de las mayores dificultades dadas por el autismo, la dificultad de establecer relaciones empáticas.
 
  Ejemplos de diferentes robots y modos de interacción... Industriales, guías de museo, asistenciales, educativos
 
 \section{Interacción con robots educativos}
 \label{sec:hri-robots-educativos}
 Ejemplos específicos de robots educativos
 
 \section{Solución para el ROBOBO}
 \label{sec:hri-solucion-robobo}
 
 La plataforma ROBOBO, cuenta con características que permiten realizar diversas maneras de interacción, el hecho de estar basado en un smartphone, pone todas las capacidades del mismo a disposición a la hora de interactuar con los usuarios. Partiendo del hardware con el que se cuenta, base motorizada \textit{ROB} y smartphone \textit{OBO} , se han definido, en una analogía a los sentidos, una serie de paquetes para dotar al ROBOBO de diversas funcionalidades de interacción.
 
 El primero de estos paquetes sería el paquete de habla. Siendo el habla, probablemente, la principal forma de interacción que se da entre los humanos, parece lógico que, teniendo la capacidad de proceso de un smartphone moderno, dotar al robot ya no solo de la capacidad de producir habla, sino de entender texto hablado es un concepto interesante. Este paquete daría al robot la posibilidad de comunicación bidireccional con los humanos, además de darle al robot cierta personalidad más \enquote{orgánica} , haciéndolo más atractivo, por ejemplo, a los niños.
 
 El segundo, de forma análoga al sentido del tacto, teniendo una \enquote{piel artificial} cómo es la pantalla táctil del teléfono, dar la capacidad al robot de sentir toques y caricias parece una idea interesante a la hora de llevar una interacción cercana con el robot.
 
 El tercero, siguiendo la analogía con los sentidos, sería el sentido del oído. La capacidad de producir y reconocer sonidos abre puertas a tipos de interacción interesantes. Por ejemplo, la capacidad de reaccionar ante sonidos fuertes como palmadas o la capacidad de reconocer notas musicales podrían permitir al robot comunicarse mediante el uso de la música, y estando el ROBOBO enfocado directamente a la educación, esta clase de interacción podría emplearse para la enseñanza musical en la educación infantil y primaria.
 
 El cuarto paquete, teniendo hoy en día todos los smartphones como mínimo una cámara, busca dotar al ROBOBO de sentido de la vista. Con este sentido, el robot podría detectar la presencia de gente y a que distancia se encuentran y de esta manera adaptar su comportamiento a su entorno, por ejemplo, alejándose si nota que alguien esta muy cerca o siguiendo con la \enquote{cara} a la gente a su alrededor. La capacidad de discernir diferentes colores también es una habilidad interesante de cara a la educación de niños muy pequeños.
 
 Por último, teniendo la capacidad de conectarse a internet del smartphone, parece interesante dotar al robot de la capacidad de comunicarse a través de la red. El correo electrónico es un sistema de comunicación muy extendido y casi obligatorio hoy en día, así que, dada la extensión del sistema, y la cantidad de usuarios con los que cuenta, parece la opción correcta para este tipo de comunicación.
 
 %%todo Partir desde la plataforma, desarrollar los diferentes métodos de interacción en función de las capacidades. En función del hardware.
  
